\documentclass[pdftex, 11pt, a4paper]{article}
\usepackage[utf8]{inputenc}
\usepackage[IL2]{fontenc}
\usepackage[left=1.5cm, top=2cm, text={18cm, 25cm}]{geometry}
\usepackage{hyperref}
\usepackage{alltt}

\newcommand{\code}{\texttt}

\begin{document}
    \title{Internetové aplikace \\
        Vytvoření webového API - Covid-19 data \\\vspace{0.5cm}
        \code{DOCUMENTATION}}
    \author{Matúš Škuta (xskuta04) \\ Patrik Németh (xnemet04)}
    \maketitle

    \begin{center}
        Note: this documentation is in English to keep it consistent with the rest of
        the project.
    \end{center}

    \section{Overview}
    This project aims to implement a web API capable of handling requests for some of
    the COVID-19 related data available at the
    \href{https://data.europa.eu/euodp/en/data/group/covid-19-coronavirus-epidemic?groups=covid-19-coronavirus-epidemic}
        {EU Open Data Portal}\footnote{Full link: \code{https://data.europa.eu/euodp/en/data/group/covid-19-coronavirus-epidemic?groups=\newline{}covid-19-coronavirus-epidemic}}.
    The specific data accessible by the API in this implementation is daily hospital bed
    occupancy by COVID-19 patients, daily ICU (intensive care unit) COVID-19 patient
    admissions, and weekly COVID-19 tesing data that includes the number of new cases,
    number of tests done, and testing and positivity rates.

    The project is divided into two parts - the server providing the API and a simple
    client web application serving as an example for the possible usage of the API.

    \subsection{Requirements}
    The server requires \emph{Node.js} version \code{14.x} with the \emph{npm} package manager, with which
    all other dependencies will be automatically installed. The example client
    application requires an up-to-date web browser. All~installation instructions
    can be found in the \code{README.md} file in the top level directory of the project.

    The automatically installed packages are NOT installed systemwide. These packages are:
    \begin{itemize}
        \item \code{cors}: used for handling cross-origin resource requests,
        \item \code{country-code-lookup}: the API uses country codes and this provides a simple
            way to decode the country codes into country names,
        \item \code{express}: used as the main web framework. Handles listening for traffic and
            easier URI routing,
        \item \code{jsdoc}: for generating code documentation,
        \item \code{node-cron}: scheduling of checks for new data,
        \item \code{reflect-metadata}: dependency of \code{tsyringe} for class decoration (every class
            is a singleton),
        \item \code{tsyringe}: correct singleton class instantiation - makes sure
            that the classes are not instantiated multiple times.
    \end{itemize}

    \section{The server}
    The web server is built on \emph{Node.js} and written in \emph{TypeScript}.
    It is divided into multiple files in the \code{/server} directory, where
    the entrypoint to the application, \code{index.ts}, is located. This contains
    a single instantiation of the web server that uses the \emph{Express} framework.
    The server itself is wrapped in the \code{Server} class in~charge of route handler
    instantiaton, as well as listening on port 3000.

    Route handlers are located in the \code{/server/route\_handlers} directory.
    The class \code{RouteHandler} instantiates all the API-call-specific route handlers
    (\code{HospitalRouteHandler} and \code{TestsRouteHandler}). These are in charge of
    specifying the structure of all the available API calls (explained in section
    \ref{sec:api_structure}) and the behaviour of how to handle these calls. As this web
    API is meant to act like a form of data acquisition tool, the only HTTP method
    supported is the \code{GET} method.

    Each of the specific route handlers contain an instance of data workers
    (found in directory \newline \code{/server/json\_data\_workers}), which implement methods for
    loading and filtering the raw JSON data from the original source. This data is loaded
    every time the server is started. If the server is not shut down for an extended period,
    it will periodically send a \code{HEAD} request to the source server and if the
    \code{Last-Modified} response field specifies a more recent date than during
    the last data download, the new data is downloaded.

    \subsection{API call structure} \label{sec:api_structure}
    The API calls supported by this web server are all for the \code{GET} HTTP method.

    \begin{alltt}
    /beds
        Hospital bed occupancies by COVID-19 patients per country.
        Source: https://data.europa.eu/euodp/en/data/dataset/
                    hospital-and-icu-admission-rates-and-occupancy-for-covid-19
        Routes:
            /<country>
            /<country>/<date>
            /<country>/<date-start>/<date-end>
        Where:
            <country>   is a country code (SK, DE, ...),
            <date>      is in the form of /<day>/<month>/<year>,
                        where each of the specified fields are integers.

    /icu
        The occupancy of intensive care units by COVID-19 patients per country.
        Source: https://data.europa.eu/euodp/en/data/dataset/
                    hospital-and-icu-admission-rates-and-occupancy-for-covid-19
        Routes:
            /<country>
            /<country>/<date>
            /<country>/<date-start>/<date-end>
        Where:
            <country>   is a country code (SK, DE, ...),
            <date>      is in the form of /<day>/<month>/<year>,
                        where each of the specified fields are integers.

    /tests
        Weekly COVID-19 test counts, new case counts, testing rates,
        and positivity rates per country.
        Source: https://data.europa.eu/euodp/en/data/dataset/
                    covid-19-testing
        Routes:
            /<country>
            /<country>/<year>
            /<country>/<year>/<week>
        Where:
            <country>   is a country code (SK, DE, ...),
            <year>      is the requested year, by which to filter data,
            <week>      is the requested week number of year, by which to filter data.
    \end{alltt}

    As an example, the URI \code{/beds/at/31/12/2020} would return hospital bed occupancy
    in Austria for the last day of 2020. The response structure is explained in
    section \ref{sec:api_response}.

    \subsection{API response structure} \label{sec:api_response}
    The server always returns data in the form of a JSON as an array of objects.
    Below are the documented objects contained in the JSON.\newline
    Objects returned by \code{/beds} and \code{/icu} calls and their derivatives
    have the following structure:
    \begin{alltt}
    \{
        country: (string),  The full country name.
        date: (string),     Date in the form "\{yyyy\}-\{mm\}-\{dd\}".
        value: (integer)    The value for that day. Meaning depends on the API call.
    \}
    \end{alltt}
    Objects returned by \code{/tests} calls and its derivatives
    have the following structure:
    \begin{alltt}
    \{
        country: (string),          The full country name.
        year_week: (string),        The year and week in the form "\{yyyy\}-W\{ww\}".
        new_cases: (integer),       New cases for that week.
        tests_done: (integer),      Tests done that week.
        population: (integer),      Population of the country.
        testing_rate: (string),     The testing rate for that week.
        positivity_rate: (string)   The test positivity rate for that week.
    \}
    \end{alltt}

    \section{The client}
    The client application can be found in the \code{/client} directory. It serves as
    an example of how the web API could be used by a third party. The application
    visualizes data received from the server, where the API request is built based on
    the user's input. The web server must be running for the client application to
    be able to fetch data.

    The application is run by simply opening the \code{index.html} file in a web browser.
    The user may then pick from two categories - \emph{Hospitals} and \emph{Tests}. These
    categories have different options. A shared option, however, is the country selection
    that specifies which country's data should be visualized.

    The \emph{Hospitals} category has options for showing hospital admissions or ICU
    admissions in a time frame. The time frame is selected by the two date selectors,
    where the first selects the beginning of the time frame and the second its end.
    If only the first is used to select a date, then every day since then until today
    is visualized. If only the second is selected, then only that one day is visualized.
    If none are selected, all the available data is visualized.

    The \emph{Tests} category has options for showing new cases, tests done, testing rates,
    and positivity rates in a time frame. The time frame resulotion in this category is
    weekly. If the year selector has a specific year selected and the week selector specifies
    the whole year, then that year is visualized. If both have specific values selected,
    then just that week is shown. If the "overall" option is selected in the year selector,
    then all available data is visualized.

\end{document}