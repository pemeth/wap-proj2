\documentclass[pdftex, 11pt, a4paper]{article}
\usepackage[utf8]{inputenc}
\usepackage[IL2]{fontenc}
\usepackage[left=1.5cm, top=2cm, text={18cm, 25cm}]{geometry}
\usepackage{hyperref}

\newcommand{\code}{\texttt}

\begin{document}
    \title{Internetové aplikace \\
        Vytvoření webového API - Covid-19 data \\\vspace{0.5cm}
        \code{DOCUMENTATION}}
    \author{Matúš Škuta (xskuta04) \\ Patrik Németh (xnemet04)}
    \maketitle

    \begin{center}
        Note: this documentation is in English to keep it consistent with the rest of
        the project.
    \end{center}

    \section{Overview}
    This project aims to implement a web API capable of handling requests for some of
    the COVID-19 related data available at the
    \href{https://data.europa.eu/euodp/en/data/group/covid-19-coronavirus-epidemic?groups=covid-19-coronavirus-epidemic}
        {EU Open Data Portal}\footnote{Full link: \code{https://data.europa.eu/euodp/en/data/group/covid-19-coronavirus-epidemic?groups=\newline{}covid-19-coronavirus-epidemic}}.
    The specific data accessible by the API in this implementation is daily hospital bed
    occupancy by COVID-19 patients, daily ICU (intensive care unit) COVID-19 patient
    admissions, and weekly COVID-19 tesing data that includes the number of new cases,
    number of tests done, and testing and positivity rates.

    The project is divided into two parts - the server providing the API and a simple
    client web application serving as an example for the possible usage of the API.

    \subsection{Requirements}
    The server requires \emph{Node.js} version \code{14.x} with the \emph{npm} package manager, with which
    all other dependencies will be automatically installed. The example client
    application requires an up-to-date web browser. All~installation instructions
    can be found in the \code{README.md} file in the top level directory of the project.

\end{document}